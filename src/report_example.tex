
\documentclass[varwidth=\maxdimen,border=20pt]{standalone}
\usepackage{tikz}
\usetikzlibrary{arrows.meta}
\usetikzlibrary{shapes}
\usetikzlibrary{backgrounds,shadows}

\begin{document}  
\begin{figure}
    \centering
    %Process: PROCESS1
    \begin{tikzpicture}[baseline, y=-0.7cm]
    
        % Declare the states
        \node[draw, ellipse] at (0, 0) (00) {$0$};
        \node[draw, ellipse] at (6, 0) (10) {$1$};
        \node[draw, ellipse] at (3, 0) (A0) {$A_0$};
        \node[draw, ellipse] at (3, 3) (A1) {$A_1$};
        \node[draw, ellipse] at (3, 6) (A2) {$A_2$};
        \node[draw, ellipse] at (3, 9) (A3) {$A_3$};
        \node[draw, ellipse] at (3, 12) (A4) {$A_4$};
    
        % Declare edges
        \draw[->] (A0) -- (00) node[midway,below] {a};
        \draw[->] (A0) -- (A1) node[midway,left] {a};
        \draw[->] (A1) -- (A2) node[midway,left] {a};
        \draw[->] (A2) -- (A3) node[midway,left] {a};
        \draw[->] (A3) -- (A4) node[midway,left] {a};
        \draw[->] (A4) -- (10) node[midway,above,right] {a};
    
        % Declare initial edges
        \draw[->] (2, -1) -- (A0) node[midway] {};
    
        % Declare final edges
        \draw[->] (10) -- (7, -1) node[midway] {};
        \draw[->] (A0) -- (4, -1) node[midway] {};
    
    \end{tikzpicture}
    \caption{The process has been given a caption.}
    \label{fig:process1}
\end{figure}
\vspace{2em}
\end{document}